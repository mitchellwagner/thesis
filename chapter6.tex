\chapter{Detector Testing}
The last step of manufacturing is to test the electrical properties of the detector and evaluate whether it is functioning properly or not.
Several properties must be analyzed in order to determine the quality of the detector; the first of which is leakage current.
In order for the detector to work properly, it must function as a basic capacitor.
If current can flow while the detector has applied voltage, then the amorphous germanium layer is not properly blocking holes and electrons which means the detector will not function.
Thus, a measure of leakage current is a necessary initial step to analyze if the detector will work at all.

If the leakage current is tested and determined to be of an acceptable level (on the order of a few picoamps or less), testing can proceed.
The next property measured is the detector capacitance.
This measurement is made using an injected pulsar and can also determine the depletion voltage of the detector.
The detector capacitance and depletion voltage can also give an accurate measurement for the impurity concentration of the germanium.
Finally the detector spectrum is measured using several radioactive sources such as cobalt 60 and cesium 137.
From this spectrum, the energy resolution of the detector can be determined and thus the overall quality.

All of these measurement must be taken while the detector is at cryogenic temperatures as described in chapter 2.
They also require the integration of several pieces of electronics that must be easily interchanged to make the different measurements.
The cryostat used at USD is shown in Figure \ref{fig:cryostat-whole}.
In the figure, the black cords coming into the base on the right and left side are a mix of high voltage and signal cables.
The flex tube coming into the middle is a vacuum tube.
The large black cylinder on top is the chicken feeder style liquid nitrogen dewar.
The metal cylinder at the base is where the detector sits.
\begin{figure}[htpb]
\centering
\includegraphics[width=\textwidth]{cryostat-whole}
\caption{Cryostat system used for detector calibration.}
\label{fig:cryostat-whole}
\end{figure}

\section{Detector Cryostat}
All germanium detector cryostats have one thing in common: they must be under vacuum.
The detector itself is mounted on some surface which is covered with a metal cap.
The air is then vacuumed from the enclosure and the detector is able to be efficiently cooled to under 100K.
The method of how the detector is cooled is what varies among different cryostat designs.

Figure \ref{fig:cryostats} shows two types of cryostat designs.
The left side of the figure depicts a cryostat with a cooling finger design.
Typically the detector is in thermal contact with a long metal rod that is submerged in liquid nitrogen.
The rod will eventually reach thermal equilibrium with the liquid nitrogen and is able to then able cool the detector.
\begin{figure}[htpb]
\centering
\includegraphics[width=\textwidth]{cryostats}
\caption{Two different cryostat designs}
\label{fig:cryostats}
\end{figure}
The right side of Figure \ref{fig:cryostats} depicts another cryostat that is gravity fed.
Liquid nitrogen is added to a Dewar that sits on top of the detector housing.
Liquid nitrogen can then fill tubes that are connected to the stage holding the detector.
The stage will then reach equilibrium with the liquid nitrogen and cool the detector.

Both designs come with advantages and disadvantages.
The cold finger design is widely used commercialy due to its simple design and adaptability to fit many Dewars.
However, the cold finger has draw backs in the minimum temperature that the cryostat can reach.
Portions of the cold finger in direct contact with LN can reach LN temperature, however, points out of the nigtrogen will have a temperature gradiant that increases along the shaft.
Thus the detector itself will usually be 20 to 30 degrees above that of liquid nitrogen.
The cooling itself is also quite slow with the cold finger method and it can take half a day to cool down.

The gravity fed design has an advantage that the liquid nitrogen fills the L shaped tube which directly connects to the detector stage.
This results in the detector being much closer to LN  and thus LN temperatures.
The detector can also be relatively quickly cooled down, reaching 80 Kelvin in approximately four hours.


As shown in Figure \ref{fig:cryostat-whole}, the cryostat used at USD for detector characterization and testing is of the gravity fed design.
This particular cryostat was custom designed and manufactured at LBNL.
It has the ability to measure and control the temperature of the germanium detector through the use of a thermistor and heater resistor.
Removing the aluminum can on the front of the detector reveals the inner detector along with the cooling structure as shown in Figure \ref{fig:inner-cryostat}.
Here the detector can be seen as the top hat shaped crystal with a gold foil covered contact placed on the top surface.
Several wires can also be seen leading to the external electronics and circuitry.
\begin{figure}[htpb]
\centering
\includegraphics[width=0.8\textwidth]{inner-cryostat}
\caption{}
\label{fig:inner-cryostat}
\end{figure}

cold finger is easier for user since all they have to do is dip into LN, however it is hard to open or not allowed.
Gravity fed design, u are responsible for vacuum, it is for reserach

Figure \ref{fig:outline} shows a view of the cryostat head alond with the attached input boxes, vacuum tube, and the interface between the cryostat and gravity fed Dewar.


\begin{figure}[htpb]
\centering
\includegraphics[width=\textwidth]{outline}
\caption{Simple circuit diagram of the germanium detector electronics}
\label{fig:outline}
\end{figure}

\begin{figure}[htpb]
\centering
\includegraphics[width=\textwidth]{crate}
\caption{The electronics crate used in detector characterization}
\label{fig:crate}
\end{figure}

\begin{figure}[htpb]
\centering
\includegraphics[width=0.7\textwidth]{left-ear}
\caption{Circuit box housing the preamplifier and leakage current assembly}
\label{fig:left-ear}
\end{figure}

\begin{figure}[htpb]
\centering
\includegraphics[width=0.7\textwidth]{right-ear}
\caption{Circuit box housing voltage and pulser input along with a high voltage filter}
\label{fig:right-ear}
\end{figure}

\begin{figure}[htpb]
\centering
\includegraphics[width=\textwidth]{temp-sens}
\caption{Temperature monitoring and control system with heater}
\label{fig:temp-sens}
\end{figure}

\begin{figure}[htpb]
\centering
\includegraphics[width=0.5\textwidth, angle=-90]{transimpedence}
\caption{Device for converting low current to voltage}
\label{fig:transimpedence}
\end{figure}

\begin{figure}[htpb]
\centering
\includegraphics[width=\textwidth]{cs-137-test}
\caption{Spectrum obtained from detector NI with Cs 137 and pulsar peaks highlighted in blue. x-axis is counts}
\label{fig:cs-137-test}
\end{figure}

\begin{figure}[htpb]
\centering
\includegraphics[width=0.8\textwidth]{662-pulsar}
\caption{Cs-137 and injected pulsar peak information with FWHM}
\label{fig:662-pulsar}
\end{figure}



Chapter Detector characterization
Sections- each with methods and results
Leakage current
Capacitance
Spectrum

