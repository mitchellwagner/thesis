\chapter{Introduction and Background}

Since birth, the Earth has been under a constant barrage of particles.
Some can be seen with the naked eye, such as a portion of the light from the sun and stars, while others such as x-rays or microwaves cannot.
The discovery of radioactivity by Henri Becquerel in 1896 when he was studying phosporescence \cite{Mould} coupled with Wilhelm R{\"o}ntgen's discovery of x-rays in 1895 \cite{wilhelm} kickstarted the search for devices capable of detecting these invisible particles.

One of the early and most widely used radiation detection device was the Geiger-M{\"u}ller tube developed in 1928 \cite{geiger}.
The device utalized a tube filled with gas coupled to a power supply and some simple electronics and was able to see several different types of ionizing radiation.
However, the drawback is that it was only able to count particles and could not give any information about the energy or type of particle observed.
Development of new radiation detectors continued in an effort to be able to precisly measure the energy carried by different types of energy and better understand their interactions with matter.
The 1950s and 60s saw the development of a new type of device called a semiconductor diode detector with the two types of material most widely used being Silicon and Germanium.
This thesis will focus on the development of germanium detectors

The initial development of germanium detectors came a few years after the concept was demonstrated using silicon.
The first working detector used a lithium drifted contact and was created in 1962 by D. V. Freck and J. Wakefield \cite{1962Natur}
How precisly these first developed detectors could measure the energy was, however, something to be desired.
They were held back by the limitations of purifying the germanium and the speed of the electronics.
This put germanium detectors on ice until further development led to the creation of high purity germanium (HPGe) detectors in 1970 \cite{Baertsch1970,Tavendale1970}.

New detectors made from HPGe were able to measure particle energy with almost 10 times betters accuracy which pushed them to be the most sensitive detectors yet created.
Since the 1970s, HPGe detector technology has been advancing rapidly along with the electronics required to make them useable.
The modern era has seen HPGE detetors used in space exploration, medicine, homeland security, natural sciences and now for the next generation of physics experiments.

HPGe detectors have been a standard in the scientific industry for the last 40 years due to their precision and accuracy measuring certain types of radiation in material assay.
More recently, National labs and companies have been producing specialized detectors for multiple cutting edge experiments in the areas of nuclear and astroparticle physics, astronomy, and the search for physics beyond the standard model.
Germanium detectors are becoming a standard tool for a broad spectrum of uses in the next generation of physics experiments and will remain on the forefront for years to come.

nuclear physics \cite{gretina} \cite{agata}

astronomy \cite{astronomy}

dark matter 
CDEX \cite{cdex}
COGENT \cite{cogent}

neutrino physics
Majorana \cite{majorana}
GERDA \cite{gerda}
COHERENT \cite{coherent}


Radioactive material such as uranium has seen use in the energy sector in the form of fuel for nuclear reactors and also in the weapons sector as the payload of nuclear bombs.
It can be dangerous material if handled improperly and in the past has led to mass destruction of cities and ecological zones.
Thankfully, dangerous materials like uranium and polonium can be uniquely identified by the radiation they give off.
Germanium detectors have the ability to detect and identify these elements and can even determine the quantity and direction they are in.
This is what makes HPGe detectors useful to government organizations like the Department of Homeland Security in the United States.
Portable devices are in service all around the US at ports and border crossings to search for the import and export of dangerous material with development ongoing at places like the Pacific Northwest National Laboratory \cite{StaveHS}.

Several applications of HPGe are also seen in the field of nuclear medicine.
Radiopharmaceuticals are dosed to a patient who can then be imaged using an external detector.
For example, single-photon emission computed tomography (SPECT) is a technique to image the gamma rays emitted by injected radiopharmaceuticals.
HPGe detectors have been succesfully demonstrated to have potential in small animal SPECT \cite{SPECT} and also with use in PET scanners.

