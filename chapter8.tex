\chapter{Conclusion}
Germanium detectors have had and will continue to have a major impact on the scientific comunity as a whole and is emerging into the fields of healthcare and defense.
It is one of the leading technologies in modern day physics experiments even though the technology is over 50 years old.
HPGe will continue to see explosive growth over the coming years as experimental needs require its continued advancement.

Currently only a handfull of companies and labs are able to create working HPGe detectors and far less have the capability to start with germanium in its raw form, refine it, pull the crystal, and then convert it into a working detector.
The University of South Dakota physics department has demonstrated over the last several years that they have the capacity to grow quality HPGe crystals and thus the next natural step was to create detectors.

Proving the capability of the cutting procedure was the first step in the direction of detector making.
If the detectors could not be properly cut from the crystal boule, no further work could happen.
After the diamond saw was set up and shown to work proficiently, it was necessary to verify the workings of the other polishing steps including lapping and chemical polishing.
This involved setting up the fume hudes, aquiring the proper abrasives and chemicals, and testing out the process.

Once the mechanical and chemical steps were reproducable, it was time to move on to testing the thin film deposition machines: sputtering and E-beam.
The sputtering machine took a bit of initial maintenence but was quickly brought up to working standards.
The E-beam machine only required a bit of cleaning and quickly started working.
The first step to using these machines involved determining the deposition rate of each.
Similar tests were devised using partially covered glass slides and a surface profiler to measure the deposition hight.
It was determined how long it was necessary to run the sputtering machine to acheive the desired thickness and the deposition rate and thickness monitor of the e-beam were verified to be within a few percent accuracy.

After all of the equipment was verified to be working and initial operating parameters were determined, it was time to do a dry run with a piece of germanium of known poor quality.
This was just to test to see if the sample could make it all the way through the process with minimal damage.
Next a high quality sample was used.
On the first try it was determined that the detector would not function properly.
However, after reseting the sample and trying again, it was shown to function as a radiation detector.

Several more working detectors have been produced since the first with increasingly good performance.
These detectors show that USD is capable of going through the entire process chain of manufacturing HPGe detectors from scratch.
