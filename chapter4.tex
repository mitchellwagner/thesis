\chapter{Manufacturing of Planar Detectors}
The manufacturing process of a planar HPGe detector begins with a slice from a crystal boule that has been tested for quality and is know to be detector grade. Typical boules slices are solid discs that can range from a few millimeters up to several centimeters in thickness and 5+ centimeters in diameter. This large size allows for several detector samples to be cut from each slice so careful geometry considerations are important in order to minimize wasted material.
% INSTERT PICTURE OF DETECTOR SHAPE 2 WINGS AND 4 WITH DIMENSIONS


\subsection{Mechanical Processing}

\subsection{Chemical Processing}

\subsection{Amorphous Ge Deposition}

\begin{sidewaysfigure}
\includegraphics[width=\textwidth]{figures/sput-flow}
\caption{This is a diagram of the Sputtering machine vacuum and gas system. Each valve is connected to a pressurized air line.}
\label{LandscapeFigure}
\end{sidewaysfigure}

\subsection{Aluminum Deposition}

\begin{sidewaysfigure}
\includegraphics[width=\textwidth]{figures/ebeam-flow}
\caption{This is a diagram of the of the electron beam machine.It is used to deposit aluminum onto the detector sample.}
\label{LandscapeFigure}
\end{sidewaysfigure}


\subsection{Final Steps}
